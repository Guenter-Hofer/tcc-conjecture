\documentclass[12pt,a4paper]{article}
\usepackage[utf8]{inputenc}
\usepackage[english]{babel}
\usepackage{amsmath}
\usepackage{amsfonts}
\usepackage{amssymb}
\usepackage{graphicx}
\usepackage{lmodern}
\usepackage[left=2cm,right=2cm,top=2cm,bottom=2cm]{geometry}
\begin{document}
\title{Conjecture about tautologies, contingencies
and contradictions}
\author{Günter Hofer}
\maketitle
\newpage 
\paragraph{Introduction}
The satisfiability problem is not easy. With n variables there are $2^n$ possibilities in propositional logic. But when you only need to know if a given Boolean formula
is satisfiable, there is a simple method.


\[A \land B := A + B - A \times B\]
\[A \lor B 		:= A + B - A \times B\]
\[\lnot A := 1 - A\]
\[A \supset B := 1 - (x \times (1 - y))\]
\[A \equiv B := (A \supset B) \times (B \supset A) \]
\[A \oplus B := 1 - (A \equiv B)\]
\[A \bar{\land} B := 1- A \times B\]
\[A \bar{\lor}B := 1 - (A + B - A \times B)\]
\paragraph{}
A and B are real numbers in [0,1] .
\paragraph{}
Here is a J program.
\begin{verbatim}
NB. logic functions as arithmetic formulas
not =: 3 : '1-y'
and =: 4 : 'x*y'
or =: 4 : 'x+y-x*y'
imp =: 4 : '1-(x*(1-y))'
eqv =: 4 : '(x imp y) * (y imp x)'
xor =: 4 : '1 - (x eqv y)'
nand =: 4 : '1-x*y'
nor =: 4 : '1-x+y-x*y'
\end{verbatim}

\paragraph{Tautology Conjecture}
If and only if all results, whereby at least one variable has at least 2 different values, are above 0.5 it can be a tautology.

\paragraph{Contradiction Conjecture}
If and only if all results, whereby at least one variable has at least 2 different values, are under 0.5 it can be a contradiction.

\paragraph{Contingency Conjecture}
If neither the Tautology Conjecture nor the Contradictions Conjecture applies, then it is satisfiable but not a tautology.
\newpage 
\paragraph{Examples}

It is assumed that the above J program is loaded.
\begin{verbatim}
  a=.b=.c=.d=.0.5
   ((a or b) and (b imp c)) or (c imp d) or (c imp b)
0.972656
   a=.0 1
   ((a or b) and (b imp c)) or (c imp d) or (c imp b)
0.960938 0.984375
   a=.a,a
   b=.0 0 1 1
   a
0 1 0 1
   b
0 0 1 1
   ((a or b) and (b imp c)) or (c imp d) or (c imp b)
0.875 1 1 1
   a=.0
   b=.0
   c=.0 1
   ((a or b) and (b imp c)) or (c imp d) or (c imp b)
1 0.5
NB. Because of the 0.5 you know it is a contingency
   c=.1
   d=. 0 1
   ((a or b) and (b imp c)) or (c imp d) or (c imp b)
0 1
\end{verbatim}

\begin{verbatim}
   a=.b=.0.5
   (a imp (b imp a))
0.875
   a=. 0 1
   (a imp (b imp a))
1 1


   a=.b=.0.5
   (a imp (b imp a))
0.875
   b=.0 1
   (a imp (b imp a))
1 0.75
   b=.1
   a=.0 1
   (a imp (b imp a))
1 1

\end{verbatim}
\end{document}
